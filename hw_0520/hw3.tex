\documentclass[extendedabs]{bmvc2k}
\usepackage[ruled]{algorithm2e}
\begin{document}

\title{Digital Image Processing HW3}
\addauthor{Hejung Yang}{}{1}
\addinstitution{School of Electrical and Electronic Engineering, 
Yonsei University.}

\maketitle
\vspace{-0.2in}

\section*{problem 1: fft and frequency spectra}

\begin{figure}[h]
    \centering
    \includegraphics[width=0.7\linewidth]{hw3_1_1}
    \caption{result of prob1.m}
    \label{fig:1}
\end{figure}

In prob1.m, each row contains the spatial domain image and its fft output.
At first row, given input image consists of half black and half white, 
applying 2-dimensional fft and shifting the zero frequency component to the center
results of right image. It can be seen from the frequency domain image that the given
image contains horizontal frequency component, which stands for vertical edge.
At second row, given input image of white square on the center, it can be seen from
the shifted 2-d fft result that the image contains both horizontal and vertical frequency
component. Rotating the second original image also affects the frequency component angle.

\section*{problem 2: phase and magnitude}

\begin{figure}[h]
    \centering
    \includegraphics[width=\linewidth]{hw3_2_1}
    \caption{result of hw3\_2.m. leftmost: original image, middle: log-scaled magnitude
    of the given image, rightmost: phase of the image}
    \label{fig:2}
\end{figure}

hw3\_2.m reads image, calculate magnitude/phase after applying 2d fast fourier transform(fft2)
onto the image, shifting the zero frequency component to the center of the image.
Given complex-valued frequency domain image $\omega$, magnitude, phase can be calculated as follows:
\[\omega = \mathcal{F}(image)\]
\[magnitude(\omega) = \sqrt{Real(\omega)^2 + Imag(\omega)^2}\]
\[phase(\omega)) = tan^{-1}\frac{Imag(\omega)}{Real(\omega)}\]
For better visualization, plotted magnitude is log-scaled.
\figurename{\ref{fig:2}} is the magnitude/phase plot of the image.
It can be seen that most of the magnitude is concentrated on the center of the image.
Also, phase image seems to have random values.

\section*{problem 3: phase vs. magnitude}

\begin{figure}[h]
    \centering
    \includegraphics[width=\linewidth]{hw3_3_1}
    \caption{result of hw3\_3.m. top-left: original mandrill image, top-right: original clown image,
    bottom-left: reconstructed image having magnitude of mandrill and phase of clown,
    bottom-right: reconstructed image having magnitude of clown and phase of mandrill}
    \label{fig:3}
\end{figure}

hw3\_3.m reads mandrill and clown image, swaps their phase information and reconstruct the image.
Given magnitude $mag$ and phase $ph$, complex-valued frequency domain image can be calculated as follows:
\[\omega = mag * (cos(ph) + sin(ph) * 1j)\]
\[image = \mathcal{F}^{-1}(\omega)\]

It can be simply seen that the new $\omega$ has magnitude $mag$ and phase $ph$:
\[magnitude(\omega) = \sqrt{Real(\omega)^2 + Imag(\omega)^2}\]
\[= \sqrt{(mag * cos(ph))^2 + (mag * sin(ph))^2}\]
\[= \sqrt{mag^2 * (cos^2(ph) + sin^2(ph))}\]
\[= \sqrt{mag^2} = mag\] 
\[phase(\omega) = tan^{-1}\frac{Imag(\omega)}{Real(\omega)}\]
\[= tan^{-1}\frac{sin(ph)}{cos(ph)}\]
\[= tan^{-1}(tan(ph)) = ph\]

Result of swapping the phase can be found in \figurename{\ref{fig:3}}.
It can be seen that the bottom-left image, consists of mandrill's magnitude and clown's phase,
looks much similar to clown than mandrill. Conversely, bottom-right image which is from clown's
magnitude and mandrill's phase looks like mandrill imge.

\begin{figure}[h]
    \centering
    \includegraphics[width=\linewidth]{hw3_3_2}
    \caption{magnitude, phase of images in \figurename{\ref{fig:3}}. Each row stands for
    top-left(mandrill), top-bottom(clown), bottom-left and bottom-right as ordered}
    \label{fig:4}
\end{figure}

\figurename{\ref{fig:4}} shows the magnitude/phase of the original/transformed images. It can
be seen that only the phase information is swapped while having same magnitude.
From the figures, it is shown that phase information is more important than magnitude in terms
of recognizing the object semantic and its structure; magnitude is applied so that the overall
image level is adjusted to the targetting image. In bottom-left of \figurename{\ref{fig:3}},
though the image is perceived as the clown image, it lost some intensity informations compared
to the original clown. 

\section*{problem 4: notch filter}

\begin{figure}[h]
    \centering
    \includegraphics[width=\linewidth]{hw3_4_1}
    \caption{result of hw3\_4.m. top row: original image and its magnitude in frequency 
    domain respectively. bottom row: after applying notch filter onto the frequency domain image
    and its resulting magnitude}
    \label{fig:5}
\end{figure}

hw3\_4.m reads pattern.tif, transform into complex-valued frequency domain, filter out grid pattern
by notch filter and inverse-transform to be able to be seen in spatial domain. 
\figurename{\ref{fig:5}} shows the result of applying notch filter onto the noisy image.
From the magnitude of the original image, it can be seen that there exists outstanding
vertial/horizontal lines which partially represent the grid pattern overlaid onto the original
image. After applying notch filter masking out the vertical/horizontal lines excluding the center,
it can be seen that the grid pattern is smoothed out.

\begin{figure}[h]
    \centering
    \includegraphics[width=\linewidth]{hw3_4_2}
    \caption{image and magnitude plot for components within the original image. top row: original image
    and its magnitude, middle row: grid-only image with its magnitude, bottom row: after
    subtracting grid from the original image and its magnitude}
    \label{fig:6}
\end{figure}

\figurename{\ref{fig:6}} contains the groundtruth grid pattern and target filtering result.
The groundtruth grid pattern is constructed from pattern.tif using public paint applications.
It can be seen from the middle row of the \figurename{\ref{fig:6}} that the grid pattern mainly consists
of vertical/horizontal elements within the magnitude perspective. Subtracting the image from the
grid pattern generates cleaner image, with having lower intensity on vertical/horizontal lines from
the view of magnitude.

\begin{figure*}[h]
    \centering
    \includegraphics[width=\linewidth]{hw3_4_3}
    \caption{comparison between the notch filter and the groundtruch grid pattern. top row: original image,
    notch filter applied when generating \figurename{\ref{fig:5}} and result after applying the filter 
    respectively. bottom row: same as above except the filter is from the groundtruch grid pattern}
    \label{fig:7}
\end{figure*}

\figurename{\ref{fig:7}} shows the difference between the notch filter which is applied to 
\figurename{\ref{fig:5}} and the groundtruth filter. It can be seen that both the notch
filter and the groundtruth filter contains vertial/horizontal line component within its
magnitude plot but there exists other frequency components contributing to the grid pattern
in the groundtruth plot whereas no other components exist in the notch filter.
Further improvement would be achieved when considering the detailed frequency pattern 
within the grid onto the notch filter.

\end{document}
